% !BIB program = biber
\providecommand{\topdir}{../..}
\documentclass[../../lecture_notes.tex]{subfiles}
\graphicspath{{\subfix{./images/}}}

\begin{document}
\section{Mathematical Appendix}
\subsection{System of linear ODEs}
The goal is to solve the linear system of ODEs given by:
\begin{equation}
  \frac{dx(t)}{dt}=B(t)+Ax(t),
  \label{ODE_system}
\end{equation}
where $x(t)$ is an $n\times 1$ vector, $A$ is an $n\times n$ matrix, $B(t)$ is an $n\times 1$ vector that does not depend on $x$ and the operator $\frac{d}{dt}$ takes the derivative of vectors component-wise.
Sometimes, the notation
\begin{equation*}
  \dot{x}(t):= \frac{dx(t)}{dt}
\end{equation*}
is more convenient and we will use both options interchangeably.

To solve \eqref{ODE_system}, we first solve the homogeneous part:
\begin{equation}
  \frac{dx(t)}{dt}=Ax(t).
  \label{ODE_homogeneous}
\end{equation}
If $A$ is diagonalizable, we can write
\begin{equation*}
  A=Q\Lambda Q^{-1},
\end{equation*}
where $\Lambda$ is a diagonal $n\times n$ matrix with the eigenvalues of $A$ on its diagonal and $Q$ is an $n\times n$ matrix whose columns are the eigenvectors of $A$.
By convention, we will always order the eigenvalues in decreasing order (the largest eigenvalue is $\Lambda_{11}$ and the corresponding eigenvector is the first column of $Q$ and so on).
If $A$ is not diagonalizable, then replace the diagonal matrix $\Lambda$ by the Jordan normal form $J$ of $A$, which can be written $J=D+P$ where $D$ is diagonal and $P$ is nilpotent (there exists $k$ such that $P^{k}=0$).
For these notes, we will stick to diagonalizable matrices.

Multiplying \eqref{ODE_homogeneous} by $Q^{-1}$ we get:
\begin{equation}
  \dot{y}(t)=\Lambda y(t),
  \label{diagonalODE}
\end{equation}
where
\begin{equation*}
  y(t):= Q^{-1}x(t).
\end{equation*}
The solution to \eqref{diagonalODE} is easy, since all equations are decoupled:
\begin{equation}
  y(t)=e^{\Lambda t}y_{0},
  \label{homogeneousSolution}
\end{equation}
where $y_{0}$ is the $n\times 1$ vector of initial conditions, with
\begin{equation*}
  y_{0}=Q^{-1}x_{0}.
\end{equation*}
The matrix exponential $e^{X}$ for a square matrix $X$ can be defined by the usual Taylor expansion of the exponential function
\begin{equation*}
  e^{X}:= \sum_{k=0}^{\infty}\frac{1}{k!}X^{k}.
\end{equation*}
Since $\Lambda$ is diagonal, the $i^{th}$ diagonal entry of $e^{\Lambda t}$ is $e^{\Lambda_{ii}t}$.
Multiplying \eqref{homogeneousSolution} by $Q$ gives the general solution to \eqref{ODE_homogeneous}:
\begin{align}
  Qy(t) &=Qe^{\Lambda t}y_{0} \notag \\
  x(t) &=Qe^{\Lambda t}Q^{-1}x_{0} \notag \\
  x(t) &=e^{At}x_{0}.
  \label{x_homogeneousSolution}
\end{align}

The general solution to the inhomogeneous equation \eqref{ODE_system} is the sum of the general solution \eqref{x_homogeneousSolution} and any particular solution to \eqref{ODE_system}.
A straightforward particular solution is:
\begin{equation*}
  S(t)=\Psi(t)\int_{0}^{t}\Psi(s)^{-1}B(s)ds,
\end{equation*}
where $\Psi(t)$ is the fundamental matrix of the homogeneous equation, normalized so that $\Psi(0)=I$:
\begin{equation*}
  \Psi(t):= e^{At}.
\end{equation*}
If $A$ is diagonalizable with $A=Q\Lambda Q^{-1}$, then $e^{At}=Qe^{\Lambda t}Q^{-1}$.
Therefore, the general solution is:
\begin{equation}
  x(t)=S(t)+e^{At}x_{0}.
  \label{ODE_system_solution}
\end{equation}
We verify that the general solution indeed satisfies \eqref{ODE_system}.
First, we compute
\begin{align*}
  \dot{S}(t) &=\dot{\Psi}(t)\int_{0}^{t}\Psi(s)^{-1}B(s)ds+\Psi(t)\Psi(t)^{-1}B(t) \\
  &=A\Psi(t)\int_{0}^{t}\Psi(s)^{-1}B(s)ds+B(t) \\
  &=AS(t)+B(t).
\end{align*}
Then we verify that \eqref{ODE_system_solution} satisfies \eqref{ODE_system}:
\begin{align*}
  \frac{dx(t)}{dt} &=\dot{S}(t)+\frac{d}{dt}\big(e^{At}x_{0}\big) \\
  &=\dot{S}(t)+Ae^{At}x_{0} \\
  &=AS(t)+B(t)+Ae^{At}x_{0} \\
  &=A\left( S(t)+e^{At}x_{0}\right) +B(t) \\
  &=Ax(t)+B(t).
\end{align*}

\subsubsection{The $2\times 2$ case}

If
\begin{equation*}
  A=\left[
  \begin{array}{cc}
    a & b \\
    c & d
  \end{array}
  \right],
\end{equation*}
a few good relations to remember are:
\begin{align*}
  \Lambda_{11},\Lambda_{22}& \text{ given by } \frac{\mathrm{tr}(A)\pm \sqrt{(\mathrm{tr}(A))^{2}-4\det(A)}}{2}, \\
  \Lambda_{11}+\Lambda_{22} &=\mathrm{tr}(A)=a+d, \\
  \Lambda_{11}-\Lambda_{22} &=\sqrt{(\mathrm{tr}(A))^{2}-4\det(A)}, \\
  \Lambda_{11}\Lambda_{22} &=\det(A)=ad-bc.
\end{align*}
The eigenvectors are:
\begin{equation*}
  \left[
  \begin{array}{c}
    \frac{b}{\Lambda_{11}-a} \\
    1
  \end{array}
  \right] \text{ and }
  \left[
  \begin{array}{c}
    \frac{b}{\Lambda_{22}-a} \\
    1
  \end{array}
  \right]
\end{equation*}
(assuming $b\ne 0$; if $b=0$ choose the corresponding alternatives, e.g., if $b=0$ and $c\ne 0$
use $\bigl[\,1\;\; (\Lambda_{ii}-d)/c\,\bigr]^T$, and if $b=c=0$ (diagonal $A$) use the standard basis vectors).
If $B$ is constant,
\begin{align*}
  S(t) &=\Psi(t)\int_{t_{start}}^{t}\Psi(s)^{-1}B\,ds \\
  &=e^{At}\int_{t_{start}}^{t}e^{-As}B\,ds \\
  &=Qe^{\Lambda t}\int_{t_{start}}^{t}\left[ e^{\Lambda s}\right]^{-1}Q^{-1}B\,ds \\
  &=Qe^{\Lambda t}\Lambda^{-1}\left( e^{-\Lambda t_{start}}-e^{-\Lambda t}\right) Q^{-1}B
\end{align*}
(assuming $\Lambda$ is invertible; if some $\Lambda_{ii}=0$, interpret the corresponding diagonal entry as $t-t_{start}$), where we have used that:
\begin{align*}
  \int_{t_{start}}^{t}\left[ e^{\Lambda s}\right]^{-1}ds &=\left[
  \begin{array}{cc}
    \int_{t_{start}}^{t}e^{-\Lambda_{11}s}ds & 0 \\
    0 & \int_{t_{start}}^{t}e^{-\Lambda_{22}s}ds
  \end{array}
  \right] \\
  &=\left[
  \begin{array}{cc}
    \frac{e^{-\Lambda_{11}t_{start}}-e^{-\Lambda_{11}t}}{\Lambda_{11}} & 0 \\
    0 & \frac{e^{-\Lambda_{22}t_{start}}-e^{-\Lambda_{22}t}}{\Lambda_{22}}
  \end{array}
  \right] \\
  &=\Lambda^{-1}\left( e^{-\Lambda t_{start}}-e^{-\Lambda t}\right).
\end{align*}
Then the solution is:
\begin{equation*}
  x(t)=Qe^{\Lambda t}\Lambda^{-1}\left( e^{-\Lambda t_{start}}-e^{-\Lambda t}\right) Q^{-1}B+Qe^{\Lambda t}Q^{-1}x_{0}.
\end{equation*}
We check with $t_{start}=0$:
\begin{align*}
  \dot{x}(t) &=\frac{d}{dt}\left[ Qe^{\Lambda t}\Lambda^{-1}\left( e^{-\Lambda t_{start}}-e^{-\Lambda t}\right) Q^{-1}B+Qe^{\Lambda t}Q^{-1}x_{0}\right] \\
  &=\frac{d}{dt}\left[ Q\Lambda^{-1}\left( e^{\Lambda t}-I\right) Q^{-1}B+Qe^{\Lambda t}Q^{-1}x_{0}\right] \\
  &=Q\Lambda^{-1}\Lambda e^{\Lambda t}Q^{-1}B+Q\Lambda e^{\Lambda t}Q^{-1}x_{0} \\
  &=Qe^{\Lambda t}Q^{-1}B+AQe^{\Lambda t}Q^{-1}x_{0} \\
  &=Qe^{\Lambda t}Q^{-1}B+A\left( Qe^{\Lambda t}Q^{-1}x_{0}\right) \\
  &=Ax(t)+B.
\end{align*}

\subsection{The maximum principle}

Consider the infinite-horizon optimal control problem
\begin{equation*}
  \max_{u(\cdot)} \int_{0}^{\infty} e^{-\rho t} f(x(t),u(t)) \, dt
\end{equation*}
subject to
\begin{equation*}
  \dot{x}(t)=\mu(t,x(t),u(t)), \quad x(0)=x_{0}, \quad u(t)\in\mathcal{U} \text{ for all } t \ge 0, \quad G(x(t),u(t))\ge 0,
\end{equation*}
where $x(t)\in\mathbb{R}^{n}$ are state variables, $u(t)\in\mathbb{R}^{s}$ are control variables, $u(\cdot)$ is a measurable control function, $\mathcal{U}\subset\mathbb{R}^{s}$ is the control set, $\rho>0$ is the discount rate, $f:\mathbb{R}^{n}\times\mathbb{R}^{s}\to\mathbb{R}$ is the instantaneous payoff, $\mu:\mathbb{R}_{+}\times\mathbb{R}^{n}\times\mathbb{R}^{s}\to\mathbb{R}^{n}$ gives the state dynamics, and $G:\mathbb{R}^{n}\times\mathbb{R}^{s}\to\mathbb{R}^{q}$ is a vector-valued function representing $q$ path constraints.

Define the current-value Hamiltonian
\begin{equation*}
  \mathcal{H}(t,x,u,\lambda,\phi)=f(x,u)+\lambda^{T}\mu(t,x,u)+\phi^{T}G(x,u),
\end{equation*}
with costate $\lambda(t)\in\mathbb{R}^{n}$ and multipliers $\phi(t)\in\mathbb{R}^{q}$, where $\phi(t)\ge 0$ componentwise.

\medskip

If an admissible pair $(x^{\ast}(\cdot), u^{\ast}(\cdot))$ solves the problem and suitable regularity conditions hold, then there exist an absolutely continuous function $\lambda(\cdot)$ and a measurable function $\phi(\cdot)$ such that, for almost all $t\ge 0$:

\begin{enumerate}
\item State equation and feasibility:
\begin{equation*}
  \dot{x}^{\ast}(t)=\mu(t,x^{\ast}(t),u^{\ast}(t)), \quad x^{\ast}(0)=x_{0}, \quad u^{\ast}(t)\in\mathcal{U}, \quad G(x^{\ast}(t),u^{\ast}(t))\ge 0.
\end{equation*}

\item Hamiltonian maximization:
\begin{equation*}
  u^{\ast}(t)\in\arg\max_{u\in\mathcal{U},\,G(x^{\ast}(t),u)\ge 0}\, \mathcal{H}\bigl(t,x^{\ast}(t),u,\lambda(t),\phi(t)\bigr),
\end{equation*}
and, if $u^{\ast}(t)$ is in the interior of the feasible control set, this yields the first-order condition $\partial \mathcal{H}/\partial u=0$ evaluated at $\bigl(t,x^{\ast}(t),u^{\ast}(t),\lambda(t),\phi(t)\bigr)$.

\item Costate dynamics in current value:
\begin{equation*}
  \dot{\lambda}(t)=\rho\,\lambda(t)-\frac{\partial \mathcal{H}}{\partial x}\bigl(t,x^{\ast}(t),u^{\ast}(t),\lambda(t),\phi(t)\bigr).
\end{equation*}

\item Complementary slackness and sign:
\begin{equation*}
  \phi_{j}(t)\ge 0, \quad G_{j}(x^{\ast}(t),u^{\ast}(t))\ge 0, \quad \phi_{j}(t)\,G_{j}(x^{\ast}(t),u^{\ast}(t))=0 \quad \text{for } j=1,\dots,q.
\end{equation*}
\end{enumerate}

It is sometimes convenient to work with present-value objects.
Define $p(t)=e^{-\rho t}\lambda(t)$ and $\psi(t)=e^{-\rho t}\phi(t)$, and the present-value Hamiltonian
\begin{equation*}
  \mathcal{H}^{pv}(t,x,u,p,\psi)=e^{-\rho t}f(x,u)+p^{T}\mu(t,x,u)+\psi^{T}G(x,u),
\end{equation*}
so that the costate equation becomes
\begin{equation*}
  \dot{p}(t)=-\frac{\partial \mathcal{H}^{pv}}{\partial x}\bigl(t,x^{\ast}(t),u^{\ast}(t),p(t),\psi(t)\bigr),
\end{equation*}
with the relations $p(t)=e^{-\rho t}\lambda(t)$ and $\psi(t)=e^{-\rho t}\phi(t)$ holding pointwise in time.

\medskip

\textbf{Transversality.} As a baseline transversality condition for discounted problems with a finite optimal value, one can impose
\begin{equation*}
  \lim_{t\to\infty}\mathcal{H}^{pv}\bigl(t,x^{\ast}(t),u^{\ast}(t),p(t),\psi(t)\bigr)=0,
\end{equation*}
that is,
\begin{align*}
  \lim_{t\to\infty}\bigl[e^{-\rho t}f(x^{\ast}(t),u^{\ast}(t))&+p(t)^{T}\mu\bigl(t,x^{\ast}(t),u^{\ast}(t)\bigr)\\
  &\quad+\psi(t)^{T}G\bigl(x^{\ast}(t),u^{\ast}(t)\bigr)\bigr]=0.
\end{align*}
On path segments where $G(\cdot)$ is non-binding, the term with $\psi$ vanishes by complementary slackness, and in unconstrained problems this reduces to
\begin{equation*}
  \lim_{t\to\infty}\bigl[e^{-\rho t}f(x^{\ast}(t),u^{\ast}(t))+p(t)^{T}\mu\bigl(t,x^{\ast}(t),u^{\ast}(t)\bigr)\bigr]=0.
\end{equation*}

A useful strengthening is obtained under additional structure.
Suppose the state space is $\mathbb{R}_{+}^{n}$ with $x^{\ast}(t)\ge 0$ componentwise for all $t$, the value function $V$ is concave and differentiable so that $\lambda(t)=\nabla V\bigl(x^{\ast}(t)\bigr)$, and the problem primitives imply that $\lambda(t)\ge 0$ componentwise (for example, if $f$ and the components of $\mu$ are non-decreasing in $x$).
Then the baseline condition implies
\begin{equation*}
  \lim_{t\to\infty} e^{-\rho t}\lambda(t)^{T}x^{\ast}(t)=\lim_{t\to\infty} p(t)^{T}x^{\ast}(t)=0,
\end{equation*}
because concavity of $V$ yields $\lambda(t)^{T}x^{\ast}(t)\le V\bigl(x^{\ast}(t)\bigr)-V(0)$ and hence $0\le p(t)^{T}x^{\ast}(t)\le e^{-\rho t}V\bigl(x^{\ast}(t)\bigr)-e^{-\rho t}V(0)$, with the right-hand side converging to zero if the total value is finite.

\medskip

\textbf{Sufficiency.} The necessary conditions above are also sufficient for optimality if the transversality condition holds, and if, for each $t$, the map $(x,u)\mapsto \mathcal{H}\bigl(t,x,u,\lambda(t),\phi(t)\bigr)$ is concave on the set $\{(x,u) \mid u \in \mathcal{U}, G(x,u) \ge 0\}$ and this set is convex.
In this case, any admissible path satisfying the conditions is globally optimal.


\subsection{Hamilton-Jacobi-Bellman equation (HJB)}

The Hamilton-Jacobi-Bellman (HJB) equation provides a sufficient condition for optimality and is the continuous-time analogue of the Bellman equation.
Consider the stochastic optimal control problem:
\begin{align*}
  V(t,x) &= \max_{u(\cdot)} \mathbb{E}_{t} \left[ \int_{t}^{\infty} e^{-\rho (s-t)} f(s, x(s), u(s)) ds \right] \\
  \text{s.t.} \quad dx(s) &= \mu(s, x(s), u(s))ds + \sigma(s, x(s), u(s))dZ_{s}, \\
  u(s) &\in \mathcal{U} \quad \text{for all } s \ge t, \\
  x(t) &= x.
\end{align*}
Here, $V(t,x)$ is the value function, representing the optimal value of the objective starting from state $x$ at time $t$.
The state variable $x(t) \in \mathbb{R}^n$, the control $u(t) \in \mathcal{U} \subset \mathbb{R}^s$, and $Z_t$ is an $m$-dimensional standard Brownian motion.

The HJB equation is derived from the Bellman principle of optimality, which states that for a small time interval $dt$:
\begin{equation*}
  V(t,x) \approx \max_{u \in \mathcal{U}} \left\{ f(t,x,u)dt + e^{-\rho dt} \mathbb{E}_t [V(t+dt, x+dx)] \right\}.
\end{equation*}
Using the approximation $e^{-\rho dt} \approx 1 - \rho dt$ and applying Itô's lemma to expand $V(t+dt, x+dx)$, we get:
\begin{equation*}
  \mathbb{E}_t[dV] = \left( \frac{\partial V}{\partial t} + (\nabla_x V)^T \mu + \frac{1}{2} \text{tr}\left( \sigma \sigma^T \nabla_x^2 V \right) \right) dt,
\end{equation*}
where $\nabla_x V$ is the gradient of $V$ with respect to $x$ (an $n \times 1$ column vector) and $\nabla_x^2 V$ is the $n \times n$ Hessian matrix of second partial derivatives.

Substituting this into the Bellman equation, rearranging, dividing by $dt$, and taking the limit as $dt \to 0$ yields the HJB equation:
\begin{equation}
  \rho V(t,x) = \max_{u \in \mathcal{U}} \left\{ f(t,x,u) + \frac{\partial V}{\partial t} + (\nabla_x V)^T \mu(t,x,u) + \frac{1}{2} \text{tr}\left( \sigma(t,x,u) \sigma(t,x,u)^T \nabla_x^2 V \right) \right\}.
  \label{eq:hjb}
\end{equation}
An easy way to remember this is through the economic intuition: the required return on the value function ($\rho V$) must equal the flow payoff ($f$) plus the expected rate of change of the value function ($E[dV]/dt$).

The expression inside the maximization is the Hamiltonian, though definitions can vary.
The term inside the trace is the variance-covariance matrix of the stochastic process.
Any path constraints, like $G(x,u) \ge 0$, can be included in the maximization over $u$.

\subsubsection{The Autonomous Case}
In many economic models, the functions $f$, $\mu$, and $\sigma$ do not depend explicitly on time $t$.
This is known as an autonomous problem.
In this case, the value function $V$ will also be independent of time, so $V(t,x) = V(x)$ and its partial derivative with respect to time is zero, $\frac{\partial V}{\partial t} = 0$.
The HJB equation \eqref{eq:hjb} then simplifies to the stationary HJB equation:
\begin{equation*}
  \rho V(x) = \max_{u \in \mathcal{U}} \left\{ f(x,u) + (\nabla_x V)^T \mu(x,u) + \frac{1}{2} \text{tr}\left( \sigma(x,u) \sigma(x,u)^T \nabla_x^2 V \right) \right\}.
\end{equation*}
This is a non-linear ordinary or partial differential equation for the value function $V(x)$.
\end{document}
