% !TEX root = fig_solow_prod_parametric.tex
% !TEX program = pdflatex

\documentclass{article}
\usepackage[paperwidth=234.058bp,paperheight=188.739bp,margin=0pt]{geometry}
\usepackage{tikz}
\usepackage{pgfplots}
\pgfplotsset{compat=1.18}

% Programmatic Solow diagram using model equations.
% All elements (curves, intersection, labels, guides) are computed
% from parameters and update automatically when parameters change.

% Page/inner chart geometry (bp units), matching fig_solow_prod.tex
\newcommand{\ChartShiftX}{19.271}
\newcommand{\ChartShiftY}{5.563}
\newcommand{\ChartW}{194.27399}
\newcommand{\ChartH}{161.39407}

% Styling constants to match line weights/dashes of current figure
\newcommand{\AxisLW}{0.3985pt}
\newcommand{\CurveLW}{0.79701pt}
\newcommand{\DashLen}{2.98883pt}
\definecolor{SolowGreen}{rgb}{0,0.55,0}
\definecolor{SolowBrownRed}{rgb}{0.7,0,0}

% Solow model parameters (change these to update figure)
% Production: y = A k^alpha
% Fitted alpha to approximate the reference curve shape
\pgfmathsetmacro{\alpha}{0.385}
\pgfmathsetmacro{\A}{1.0}

% Target anchor points from current figure (inner-axis coordinates)
\pgfmathsetmacro{\kstarTarget}{56.29872}
\pgfmathsetmacro{\ystarTarget}{63.4818}
\pgfmathsetmacro{\ysfTarget}{47.189}

% Set s and deltaParam to match targets programmatically
\pgfmathsetmacro{\s}{\ysfTarget/\ystarTarget}
\pgfmathsetmacro{\deltaParam}{\ysfTarget/\kstarTarget}

% Derived steady state and values in model-space
% k* solves s A k^alpha = deltaParam k => k* = (s A / deltaParam)^(1/(1-alpha))
\pgfmathsetmacro{\kstar}{pow(\s*\A/\deltaParam, 1/(1-\alpha))}
\pgfmathsetmacro{\ystar}{\A*pow(\kstar,\alpha)}
\pgfmathsetmacro{\ysf}{\s*\ystar}

% Axis scaling: we treat axis data units as bp to match fig_solow_prod
% Choose scaling so x(k*) and y(ystar) hit the same bp coordinates as the current figure
% Compute scales to map model units to bp units
% xPix = k * xscale ; yPix = y * yscale
\pgfmathsetmacro{\xscale}{\kstarTarget/\kstar}
\pgfmathsetmacro{\yscale}{\ystarTarget/\ystar}

% With these, deltaParam implied in pixels equals (ysfTarget/kstarTarget)
% but we keep model-space deltaParam and s; the scaling maps the line appropriately.

% Curves in pixel-space (bp) as PGF macros for pgfplots
\pgfmathdeclarefunction{fprod}{1}{% y = A k^alpha
  \pgfmathparse{\yscale*\A*pow(#1/\xscale,\alpha)}%
}
\pgfmathdeclarefunction{sff}{1}{% y = s * A k^alpha
  \pgfmathparse{\yscale*\s*\A*pow(#1/\xscale,\alpha)}%
}
\pgfmathdeclarefunction{deltak}{1}{% y = deltaParam * k
  \pgfmathparse{\deltaParam*(#1)}%
}

\begin{document}
\begin{tikzpicture}[x=1bp,y=1bp]
  % Place the pgfplots axis exactly at the chart origin used in fig_solow_prod
  \begin{axis}[
      scale only axis,
      width=\ChartW bp,
      height=\ChartH bp,
      at={(\ChartShiftX bp,\ChartShiftY bp)},
      anchor=south west,
      xmin=0, xmax=\ChartW,
      ymin=0, ymax=\ChartH,
      axis lines=none,
      tick style={draw=none},
      xtick=\empty, ytick=\empty,
      clip mode=individual,
    ]

    % Programmatic computation of steady state position
    \pgfmathsetmacro{\kstarPix}{\kstar*\xscale}
    \pgfmathsetmacro{\ystarPix}{\ystar*\yscale}
    \pgfmathsetmacro{\ysfPix}{\ysf*\yscale}

    % Define shortened axis endpoints (leave room for titles)
    \pgfmathsetmacro{\AxisEndX}{0.90*\ChartW}
    \pgfmathsetmacro{\AxisEndY}{0.90*\ChartH}

    % Draw axes at origin without arrows (shortened)
    \draw[line width=\AxisLW, line cap=butt]
      (axis cs:0,0) -- (axis cs:\AxisEndX,0);
    \draw[line width=\AxisLW, line cap=butt]
      (axis cs:0,0) -- (axis cs:0,\AxisEndY);

    % Define curve end point (80% of axis length to leave room for labels)
    \pgfmathsetmacro{\curveEnd}{0.80*\AxisEndX}

    % f(k): green concave curve starting from origin
    \addplot [domain=0:\curveEnd, samples=900, no marks, line width=\CurveLW, color=SolowGreen]
      ({x}, {fprod(x)});

    % s f(k): black curve starting from origin
    \addplot [domain=0:\curveEnd, samples=900, no marks, line width=\CurveLW, color=black]
      ({x}, {sff(x)});

    % δk line passing through origin
    \addplot [domain=0:\curveEnd, samples=2, no marks, line width=\CurveLW, color=SolowBrownRed]
      ({x}, {deltak(x)});

    % Dashed guides at steady state (k*, y*)
    \draw[line width=\AxisLW, line cap=butt, line join=miter, dash pattern=on \DashLen off \DashLen]
      (axis cs:0,\ystarPix) -- (axis cs:\kstarPix,\ystarPix) -- (axis cs:\kstarPix,0);

    % Tick marks on axes at k* and y*
    \draw[line width=\AxisLW] (axis cs:\kstarPix,-2) -- (axis cs:\kstarPix,2);
    \draw[line width=\AxisLW] (axis cs:-2,\ystarPix) -- (axis cs:2,\ystarPix);

    % Dots: moderate size for visibility, same size for both
    \pgfmathsetmacro{\dotRadius}{1.8pt}
    % (k*, y*) on f(k) curve
    \draw[line width=\AxisLW, fill=black] (axis cs:\kstarPix,\ystarPix) circle[radius=\dotRadius];
    % point on sf(k) at k* (filled and stroked to match)
    \draw[line width=\AxisLW, fill=black] (axis cs:\kstarPix,\ysfPix) circle[radius=\dotRadius];

    % Calculate y-values at curve end point for label positioning
    \pgfmathsetmacro{\fEnd}{fprod(\curveEnd)}
    \pgfmathsetmacro{\sfEnd}{sff(\curveEnd)}
    \pgfmathsetmacro{\deltaEnd}{deltak(\curveEnd)}

    % Curve labels positioned to the right of where each curve ends
    \node[anchor=west, color=SolowGreen] at (axis cs:\curveEnd+3,\fEnd) {$f(k_t)$};
    \node[anchor=west, color=black] at (axis cs:\curveEnd+3,\sfEnd) {$sf(k_t)$};
    \node[anchor=west, color=SolowBrownRed] at (axis cs:\curveEnd+3,\deltaEnd) {$\delta k_t$};

    % Tick labels on axes (positioned well inside to avoid clipping)
    \node[anchor=north] at (axis cs:\kstarPix,3) {$k^{*}$};
    \node[anchor=east] at (axis cs:3,\ystarPix) {$y^{*}$};

    % Axis titles: positioned at the ends of shortened axes
    % X-axis title: to the right of the x-axis
    \node[anchor=west] at (axis cs:\AxisEndX,0) {$k_t$};
    % Y-axis title: above the y-axis
    \node[anchor=south] at (axis cs:0,\AxisEndY) {$f(k_t)$};

  \end{axis}
\end{tikzpicture}
\end{document}
