% !BIB program = biber
\providecommand{\topdir}{../..} 
\documentclass[\topdir/lecture\_notes.tex]{subfiles}
\graphicspath{{\subfix{./images/}}}

\begin{document}
\captionsetup{singlelinecheck=false}

\section{First-Generation Models of Endogenous Growth}
In the Solow--Swan and neoclassical models, economic growth is generated by exogenous technological progress. 

In this section, we study the first generation of models of endogenous growth. The first of these incorporates physical and human capital accumulation. The second is Paul Romer's (1986) path--breaking article. In many ways, Romer's paper started the endogenous growth literature and rejuvenated the interest in economic growth among economists. While Romer's objective was to model "technological change," he achieved this by introducing technological spillovers. Consequently, while the competitive equilibrium of Romer's model is not Pareto optimal and the engine of economic growth can be interpreted as a form "knowledge accumulation," in many ways the model is still neoclassical in nature.

\subsection{The AK Model Revisited}
Let us start with the simplest neoclassical model of sustained growth, which we already encountered in the context of the Solow growth model, in particular, Proposition 2.10 in subsection 2.5.1. This is the so-called $AK$ model, where the production technology is linear in capital. We will also see that in fact what matters is that the accumulation technology is linear, not necessarily the production technology. But for now it makes sense to start with the simpler case of the $AK$ economy.

\subsubsection{Demographics, Preferences and Technology}
Our focus in this chapter and the next part of the book is on economic growth, and as a first pass, we will focus on balanced economic growth, defined as a growth path consistent with the Kaldor facts (recall Chapter 2). As demonstrated in Chapter 8, balanced growth forces us to adopt the standard CRRA preferences as in the canonical neoclassical growth model (to ensure a constant intertemporal elasticity of substitution).

Throughout this chapter, we assume that the economy admits an infinitely-lived representative household, with household size growing at the exponential rate $n$. The preferences of the representative household at time $t=0$ are given by
\[
U=\int_{0}^{\infty} \exp(-(\rhoRate-n) t)\left[\frac{c(t)^{1-\theta}-1}{1-\theta}\right] dt
\]
Labor is supplied inelastically. The flow budget constraint facing the household can be written as
\[
\dot{a}(t)=(\rnet(t)-n) a(t)+\wage(t)-c(t)
\]
where $a(t)$ denotes assets per capita at time $t$, $\rnet(t)$ is the interest rate, $\wage(t)$ is the wage rate per capita, and $n$ is the growth rate of population. As usual, we also need to impose the no-Ponzi game constraint:
\[
\lim _{t \rightarrow \infty}\left\{a(t) \exp \left[-\int_{0}^{t}[r(s)-n] d s\right]\right\} \geq 0
\]

The Euler equation for the representative household is the same as before and implies the following rate of consumption growth per capita:
\[
\frac{\dot{c}(t)}{c(t)}=\frac{1}{\theta}(\rnet(t)-\rhoRate)
\]
The other necessary condition for optimality of the consumer's plans is the transversality condition,
\[
\lim _{t \rightarrow \infty}\left\{a(t) \exp \left[-\int_{0}^{t}[r(s)-n] d s\right]\right\}=0
\]
As before, the problem of the consumer is concave, thus any solution to these necessary conditions is in fact an optimal plan.

The final good sector is similar to before, except that Assumptions 1 and 2 are not satisfied. More specifically, we adopt the following aggregate production function:
\[
Y(t)=A K(t),
\]
with  A>0$. Notice that this production function does not depend on labor, thus wage earnings, $\wage(t)$, in (11.2) will be equal to zero. This is one of the unattractive features of the baseline $AK$ model, but will be relaxed below (and it is also relaxed in Exercises 11.3 and 11.4). Dividing both sides of this equation by $L(t)$, and as usual, defining $k(t) \equiv K(t) / L(t)$ as the capital-labor ratio,$ we obtain $per capita output as
\[
\begin{aligned}
y(t) & \equiv \frac{Y(t)}{L(t)} \\
& =A k(t)
\end{aligned}
\]
Equation (11.6) has a number of notable differences from our standard production function satisfying Assumptions 1 and 2. First, output is only a function of capital, and there are no diminishing returns (i.e., it is no longer the case that $f''(\cdot)<0$ ). We will see that this feature is only for simplicity and introducing diminishing returns to capital does not affect the main results in this section (see Exercise 11.4). The more important assumption is that the Inada conditions embedded in Assumption 2 are no longer satisfied. In particular,
\[
\lim _{k \rightarrow \infty} f^{\prime}(k)=A>0
\]
This feature is essential for sustained growth.

The conditions for profit-maximization are similar to before, and require that the marginal product of capital be equal to the rental price of capital, $R(t)=\rnet(t)+\delta$. Since, as is obvious from equation (11.6), the marginal product of capital is constant and equal to $A$, thus $R(t)=A$ for all $t$, which implies that the net rate of return on the savings is constant and equal to:
\[
\rnet(t)=r=A-\delta, \text{ for all } t
\]
Since the marginal product of labor is zero, the wage rate, $\wage(t)$, is zero as noted above.

\subsubsection{Equilibrium}
A competitive equilibrium of this economy consists of paths of per capita consumption, capital-labor ratio, wage rates and rental rates of capital, $[c(t), k(t), \wage(t), R(t)]_{t=0}^{\infty}$, such that the representative household maximizes (11.1) subject to (11.2) and (11.3) given initial capital-labor ratio $k(0)$ and factor prices $[\wage(t), \rnet(t)]_{t=0}^{\infty}$ such that $\wage(t)=0$ for all $t$, and $\rnet(t)$ is given by (11.7).

To characterize the equilibrium, we again note that $a(t)=k(t)$. Next using the fact that $r=A-\delta$ and $w=0$, equations (11.2), (11.4), and (11.5) imply
\[
\begin{gathered}
\dot{k}(t)=(A-\delta-n) k(t)-c(t) \\
\frac{\dot{c}(t)}{c(t)}=\frac{1}{\theta}(A-\delta-\rhoRate) \\
\lim _{t \rightarrow \infty} k(t) \exp (-(A-\delta-n) t)=0
\end{gathered}
\]
The important result immediately follows from equation (11.9). Since the right-hand side of this equation is constant, there must be a constant rate of consumption growth (as long as $A-\delta-\rhoRate > 0$ ). The rate of growth of consumption is$ therefore $independent of the level of capital stock per person, $k(t)$. This will also imply that there are no transitional dynamics in this model. Starting from any $k(0)$, consumption per capita (and as we will see, the capital-labor ratio) will immediately start growing at a constant rate. To develop this point, let us integrate equation (11.9) starting from some initial level of consumption $c(0)$, which as usual is still to be determined later (from the lifetime budget constraint). This gives

\[
c(t)=c(0) \exp \left(\frac{1}{\theta}(A-\delta-\rhoRate) t\right)
\]

Since there is growth in this economy, we have to ensure that the transversality condition is satisfied (i.e., that lifetime utility is bounded away from infinity), and also we want to ensure positive growth (the condition $A-\delta-\rhoRate > 0$ mentioned above). We$ therefore $impose:

\[
A>\rho+\delta>(1-\theta)(A-\dep)+\theta n+\delta
\]

The first part of this condition ensures that there will be positive consumption growth, while the second part is the analog to the condition that $\rho+\theta g>g+n$ in the neoclassical growth model with technological progress, which was imposed to ensure bounded utility (and thus was used in proving that the transversality condition was satisfied).

\subsubsection{Equilibrium Characterization}
We first establish that there are no transitional dynamics in this economy. In particular, we will show that not only the growth rate of consumption, but the growth rates of capital and output are also constant at all points in time, and equal the growth rate of consumption given in equation (11.9).

To do this, let us substitute for $c(t)$ from equation (11.11) into equation (11.8), which yields

\[
\dot{k}(t)=(A-\delta-n) k(t)-c(0) \exp \left(\frac{1}{\theta}(A-\delta-\rhoRate) t\right),
\]

which is a first-order, non-autonomous linear differential equation in $k(t)$. This type of equation can be solved easily. In particular recall that if
\[
\dot{z}(t)=a z(t)+b(t)
\]
then, the solution is
\[
z(t)=z_{0} \exp (a t)+\exp (a t) \int_{0}^{t} \exp (-a s) b(s) d s
\]
for some constant  z_0$ chosen to satisfy the boundary conditions. Therefore, equation (11.13) solves for:
\[k(t)=\left\{\kappa \exp ((A-\delta-n) t)+\left[(A-\dep)(\theta-1) \theta^{-1}+\rho \theta^{-1}-n\right]^{-1}\left[c(0) \exp \left(\theta^{-1}(A-\delta-\rhoRate) t\right)\right]\right\},\]
where $\kappa$ is a constant to be determined. Assumption (11.12) ensures that
\[
(A-\dep)(\theta-1) \theta^{-1}+\rho \theta^{-1}-n>0
\]
From (11.14), it may look like capital is not growing at a constant rate, since it is the sum of two components growing at different rates. However, this is where the transversality condition becomes useful. Let us substitute from (11.14) into the transversality condition, (11.10), which yields
\[\lim _{t \rightarrow \infty}\left[\kappa+\left[(A-\dep)(\theta-1) \theta^{-1}+\rho \theta^{-1}-n\right]^{-1} c(0) \exp \left(-(A-\dep)(\theta-1) \theta^{-1}+\rho \theta^{-1}-n\right) t\right]=0.\]
Since $(A-\dep)(\theta-1) \theta^{-1}+\rho \theta^{-1}-n>0$, the second term in this expression converges to zero as $t \rightarrow \infty$. But the first term is a constant. thus the transversality condition can only be satisfied if $\kappa=0$.$ therefore $we have from (11.14) that:
\[
\begin{aligned}
k(t) & =\left[(A-\dep)(\theta-1) \theta^{-1}+\rho \theta^{-1}-n\right]^{-1}\left[c(0) \exp \left(\theta^{-1}(A-\delta-\rhoRate) t\right)\right] \\
& =k(0) \exp \left(\theta^{-1}(A-\delta-\rhoRate) t\right)
\end{aligned}
\] where the second line immediately follows from the fact that the boundary condition has to hold for capital at $t=0$. This equation naturally implies that capital and output grow at the same rate as consumption.

It also pins down the initial level of consumption as
\[
c(0)=\left[(A-\dep)(\theta-1) \theta^{-1}+\rho \theta^{-1}-n\right] k(0) .
\]
Note also that in this simple $AK$ model, growth is not only sustained, but it is also endogenous in the sense of being affected by underlying parameters. For example, consider an increase in the rate of discount, $\rho$. Recall that in the Ramsey model, this only influenced the\\
level of income per capita - it could have no effect on the growth rate, which was determined by the exogenous labor-augmenting rate of technological progress. Here, is straightforward to verify that an increase in the discount rate, $\rho$, will reduce the growth rate, because it will make consumers less patient and will$ therefore $reduce the rate of capital accumulation. Since capital accumulation is the engine of growth, the equilibrium rate of growth will decline. Similarly, changes in $A$ and $\theta$ affect the levels and growth rates of consumption, capital and output.

Finally, we can calculate the saving rate in this economy. It is defined as total investment (which is equal to increase in capital plus replacement investment) divided by output. Consequently, the saving rate is constant and given by
\[
\begin{aligned}
s & =\frac{\dot{K}(t)+\dep K(t)}{Y(t)} \\
& =\frac{\dot{k}(t) / k(t)+n+\delta}{A} \\
& =\frac{A-\rho+\theta n+(\theta-1) \delta}{\theta A},
\end{aligned}
\] where the last equality exploited the fact that $\dot{k}(t) / k(t)=(A-\delta-\rhoRate) / \theta$. This equation implies that the saving rate, which was taken as constant and exogenous in the basic Solow model, is again constant, but is now a function of parameters, and more specifically of exactly the same parameters that determine the equilibrium growth rate of the economy.

Summarizing, we have:
\begin{proposition}
Consider the above-described AK economy, with a representative household with preferences given by (11.1), and the production technology given by (11.6). Suppose that condition (11.12) holds. Then, there exists a unique equilibrium path in which consumption, capital and output all grow at the same rate $g^{*} \equiv(A-\delta-\rhoRate) / \theta>0$ starting from any initial positive capital stock per worker $k(0)$, and the saving rate is endogenously determined by (11.17).
\end{proposition}

One important implication of the $AK$ model is that since all markets are competitive, there is a representative household, and there are no externalities, the competitive equilibrium will be Pareto optimal. This can be proved either using First Welfare Theorem type reasoning, or by directly constructing the optimal growth solution.

\begin{proposition}
Consider the above-described $AK$ economy, with a representative household with preferences given by (11.1), and the production technology given by (11.6). Suppose that condition (11.12) holds. Then, the unique competitive equilibrium is Pareto optimal.
\end{proposition}
\begin{proof}
See Exercise 11.2
\end{proof}

\subsubsection{The Role of Policy}
It is straightforward to incorporate policy differences in to this framework and investigate their implications on the equilibrium growth rate. The simplest and arguably one of the most relevant classes of policies are, as also discussed above, those affecting the rate of return to accumulation. In particular, suppose that there is an effective tax rate of $\tau$ on the rate of return from capital income, so that the flow budget constraint of the representative household becomes:
\[
\dot{a}(t)=((1-\tau) \rnet(t)-n) a(t)+\wage(t)-c(t)
\]
Repeating the analysis above immediately implies that this will adversely affect the growth rate of the economy, which will now become (see Exercise 11.5):
\[
g=\frac{(1-\tau)(A-\dep)-\rho}{\theta}
\]
Moreover, it can be calculated that the saving rate will now be
\[
s=\frac{(1-\tau) A-\rho+\theta n-(1-\tau-\theta) \delta}{\theta A}
\]
which is a decreasing function of $\tau$ if $A-\delta>0$. Therefore, in this model, the equilibrium saving rate is constant as in the basic Solow model, but in contrast to that model, it responds endogenously to policy. In addition, the fact that the saving rate is constant implies that differences in policies will lead to permanent differences in the rate of capital accumulation. This observation has a very important implication. While in the baseline neoclassical growth model, even reasonably large differences in distortions (for example, eightfold differences in $\tau$ ) could only have limited effects on differences in income per capita, here even small differences in $\tau$ can have very large effects. In particular, consider two economies, with respective (constant) tax rates on capital income $\tau$ and $\tau'>\tau$, and exactly the same technology and preferences otherwise. It is straightforward to verify that for any $\tau'>\tau$,
\[
\lim _{t \rightarrow \infty} \frac{Y\left(\tau^{\prime}, t\right)}{Y(\tau, t)}=0
\] where $Y(\tau, t)$ denotes aggregate output in the economy with tax $\tau$ at time $t$. Therefore, even small policy differences can have very large effects in the long run. So why does the literature focus on the inability of the standard neoclassical growth model to generate large differences rather than the possibility that the $AK$ model can generate arbitrarily large differences? The reason is twofold: first, for the reasons already discussed, the $AK$ model, with no diminishing returns and the share of capital in national income asymptoting to 1 , is not viewed as a good approximation to reality. Second, and related to our discussion in Chapter 1, most economists believe that the relative stability of the world income distribution in the post-war era makes it more attractive to focus on models in which there is a stationary world income distribution, rather than models in which small policy differences can lead to permanent growth differences. Whether this last belief is justified is, in part, an empirical question.

\subsection{The AK Model with Physical and Human Capital}
As pointed out in the previous section, a major shortcoming of the baseline $AK$ model is that the share of capital accruing to national income is equal to 1 (or limits to 1 as in the variant of the $AK$ model studied in Exercises 11.3 and 11.4). One way of enriching the $AK$ model and avoiding these problems is to include both physical and human capital. We now briefly discuss this extension. Suppose the economy admits a representative household with preferences given by (11.1). The production side of the economies represented by the aggregate production function
\[
Y(t)=F(K(t), H(t))
\] where $H(t)$ denotes efficiency units of labor (or human capital), which will be accumulated in the same way as physical capital. We assume that the production function $F(\cdot, \cdot)$ now satisfies our standard assumptions, Assumptions 1 and 2.

Suppose that the budget constraint of the representative household is given by
\[
\dot{a}(t)=(\rnet(t)-n) a(t)+\wage(t) h(t)-c(t)-i_{h}(t)
\] where $h(t)$ denotes the effective units of labor (human capital) on the representative household, $\wage(t)$ is wage rate per unit of human capital, and $i_h(t)$ is investment in human capital. The human capital of the representative household evolves according to the differential equation:
\[
\dot{h}(t)=i_{h}(t)-\delta_{h} h(t),
\] where $\delta_h$ is the depreciation rate of human capital. The evolution of the capital stock is again given from the observation that $k(t)=a(t)$, and we now denote the depreciation rate of physical capital by $\dep$ to avoid confusion with $\delta_h$. In this model, the representative household maximizes its utility by choosing the paths of consumption, human capital investments and asset holdings. Competitive factor markets imply that
\[
R(t)=f^{\prime}(k(t)) \text { and } \wage(t)=f(k(t))-k(t) f^{\prime}(k(t))
\]
where, now, the effective capital-labor ratio is given by dividing the capital stock by the stock of human capital in the economy,
\[
k(t) \equiv \frac{K(t)}{H(t)}
\]

A competitive equilibrium of this economy consists of paths of per capita consumption, capital-labor ratio, wage rates and rental rates of capital, $[c(t), k(t), \wage(t), R(t)]_{t=0}^{\infty}$,$ such that $the representative household maximizes (11.1) subject to (11.3), (11.22) and (11.23) given initial effective capital-labor ratio $k(0)$ and factor prices $[\wage(t), R(t)]_{t=0}^{\infty}$ that satisfy (11.24).

To characterize the competitive equilibrium, let us first set up at the current-value Hamiltonian for the representative household with costate variables $\mu_a$ and $\mu_h$ :

\[
\begin{aligned}
\mathcal{H}\left(a, h, c, i_{h}, \mu_{a}, \mu_{k}\right)= & \frac{c(t)^{1-\theta}-1}{1-\theta}+\mu_{a}(t)\left[(\rnet(t)-n) a(t)+\wage(t) h(t)-c(t)-i_{h}(t)\right] \\
& +\mu_{h}(t)\left[i_{h}(t)-\delta_{h} h(t)\right]
\end{aligned}
\]

Now the necessary conditions of this optimization problem imply the following (see Exercise 11.8):

\[
\begin{aligned}
\mu_{a}(t) & =\mu_{h}(t)=\mu(t) \text { for all } t \\
\wage(t)-\delta_{h} & =\rnet(t)-n \text { for all } t \\
\frac{\dot{c}(t)}{c(t)} & =\frac{1}{\theta}(\rnet(t)-\rhoRate) \text{ for all } t
\end{aligned}
\]

Combining these with (11.24),$ we obtain $that

\[
f^{\prime}(k(t))-\delta_{k}-n=f(k(t))-k(t) f^{\prime}(k(t))-\delta_{h} \text {$$ for all $$} t .
\]

Since the left-hand side is decreasing in $k(t)$, while the right-hand side is increasing, this implies that the effective capital-labor ratio must satisfy

\[
k(t)=k^{*} \text{ for all } t
\]

We can then prove the following proposition:
\begin{proposition}
Consider the above-described AK economy with physical and human capital, with a representative household with preferences given by (11.1), and the production technology given by (11.21). Let $k^*$ be given by

\[
f^{\prime}\left(k^{*}\right)-\delta_{k}-n=f\left(k^{*}\right)-k^{*} f^{\prime}\left(k^{*}\right)-\delta_{h}
\]

Suppose that $f'\left(k^*\right)>\rho+\dep>(1-\theta)\left(f'\left(k^*\right)-\delta\right)+n \theta+\dep$. Then, in this economy there exists a unique equilibrium path in which consumption, capital and output all grow at the same rate $g^* \equiv\left(f'\left(k^*\right)-\dep-\rho\right) / \theta>0$ starting from any initial conditions, where $k^*$ is given by (11.26). The share of capital in national income is constant at all times.
\end{proposition}
\begin{proof}
See Exercise 11.9
\end{proof}
The advantage of the economy studied here, especially as compared to the baseline $AK$ model is that, it generates a stable factor distribution of income, with a significant fraction of national income accruing to labor as rewards to human capital. Consequently, the current model cannot be criticized on the basis of generating counter-factual results on the capital share of GDP. A similar analysis to that in the previous section also shows that the current model generates long-run growth rate differences from small policy differences. Therefore, it can account for arbitrarily large differences in income per capita across countries. Nevertheless, it would do so partly by generating large human capital differences across countries. As\\
such, the empirical mechanism through which these large cross-country income differences are generated may again not fit with the empirical patterns discussed in Chapter 3. Moreover, given substantial differences in policies across economies in the postwar period, like the baseline $AK$ economy, the current model would suggest significant changes in the world income distribution, whereas the evidence in Chapter 1 points to a relatively stable postwar world income distribution.

\subsection{The Two-Sector AK Model}
The models studied in the previous two sections are attractive in many respects; they generate sustained growth, and the equilibrium growth rate responds to policy, to underlying preferences and to technology. Moreover, these are very close cousins of the neoclassical model. In fact, as argued there, the endogenous growth equilibrium is Pareto optimal.

One unattractive feature of the baseline $AK$ model is that all of national income accrues to capital. Essentially, it is a one-sector model with only capital as the factor of production. This makes it difficult to apply this model to real world situations. The model in the previous section avoids this problem, but at some level it does so by creating another factor of production that accumulates linearly, so that the equilibrium structure is again equivalent to the one-sector $AK$ economy. Therefore, in some deep sense, the economies of both sections are one-sector models. More important than this one-sector property, these models potentially blur key underlying characteristic driving growth in these environments. What is important is not that the production technology is $AK$, but the related feature that the accumulation technology is linear. In this section, we will discuss a richer two-sector model of neoclassical endogenous growth, based on Rebelo's (1991) work. This model will generate constant factor shares in national income without introducing human capital accumulation. Perhaps more importantly, it will illustrate the role of differences in the capital intensity of the production functions of consumption and investment.

The preference and demographics are the same as in the model of the previous section, in particular, equations (11.1)-(11.5) apply as before (but with a slightly different interpretation for the interest rate in (11.4) as will be discussed below). Moreover, to simplify the analysis, suppose that there is no population growth, i.e., $n=0$, and that the total amount of labor in the economy, $L$, is supplied inelastically.

The main difference is in the production technology. Rather than a single good used for consumption and investment, we now envisage an economy with two sectors. Sector 1 produces consumption goods with the following technology

\[
C(t)=B\left(K_{C}(t)\right)^{\alpha} L_{C}(t)^{1-\alpha}
\]
 where the subscript " $C$ " denotes that these are capital and labor used in the consumption sector, which has a Cobb-Douglas technology. In fact, the Cobb-Douglas assumption here is quite important in ensuring that the share of capital in national income is constant (see Exercise 11.12). The capital accumulation equation is given by:

\[
\dot{K}(t)=I(t)-\dep K(t)
\]
 where $I(t)$ denotes investment. Investment goods are produced with a different technology than (11.27), however. In particular, we have

\[
I(t)=A K_{I}(t)
\]

The distinctive feature of the technology for the investment goods sector, (11.28), is that it is linear in the capital stock and does not feature labor. This is an extreme version of an assumption often made in two-sector models, that the investment-good sector is more capital-intensive than the consumption-good sector. In the data, there seems to be some support for this, though the capital intensities of many sectors have been changing over time as the nature of consumption and investment goods has changed.

Market clearing implies:

\[
K_{C}(t)+K_{I}(t) \leq K(t)
\]

for capital, and

\[
L_{C}(t) \leq L
\]

for labor (since labor is only used in the consumption sector).\\
An equilibrium in this economy is defined similarly to that in the neoclassical economy, but also features an allocation decision of capital between the two sectors. Moreover, since the two sectors are producing two different goods, consumption and investment goods, there will be a relative price between the two sectors which will adjust endogenously.

Since both market clearing conditions will hold as equalities (the marginal product of both factors is always positive), we can simplify notation by letting $\kappa(t)$ denote the share of capital used in the investment sector

\[
K_{C}(t)=(1-\kappa(t)) K(t) \text { and } K_{I}(t)=\kappa(t) K(t)
\]

From profit maximization, the rate of return to capital has to be the same when it is employed in the two sectors. Let the price of the investment good be denoted by $p_I(t)$ and that of the consumption good by $p_C(t)$, then we have

\[
p_{I}(t) A=p_{C}(t) \alpha B\left(\frac{L}{(1-\kappa(t)) K(t)}\right)^{1-\alpha}
\]

Define a steady-state (a balanced growth path) as an equilibrium path in which $\kappa(t)$ is constant and equal to some $\kappa \in[0,1]$. Moreover, let us choose the consumption good as the\\
numeraire, so that $p_C(t)=1$ for all $t$. Then differentiating (11.29) implies that at the steady state:

\[
\frac{\dot{p}_{I}(t)}{p_{I}(t)}=-(1-\alpha) g_{K}
\]
 where $g_K$ is the steady-state ( BGP ) growth rate of capital.\\
As noted above, the Euler equation for consumers, (11.4), still holds, but the relevant interest rate has to be for consumption-denominated loans, denoted by $r_C(t)$. In other words, it is the interest rate that measures how many units of consumption good an individual will receive tomorrow by giving up one unit of consumption today. Since the relative price of consumption goods and investment goods is changing over time, the proper calculation goes as follows. By giving up one unit of consumption, the individual will buy $1 / p_{I}(t)$ units of capital goods. This will have an instantaneous return of $r_{I}(t)$. In addition, the individual will get back the one unit of capital, which has now experienced a change in its price of $\dot{p}_{I}(t) / p_{I}(t)$, and finally, he will have to buy consumption goods, whose prices changed by $\dot{p}_{C}(t) / p_{C}(t)$. Therefore, the general formula of the rate of return denominated in consumption goods in terms of the rate of return denominated in investment goods is

\[
r_{C}(t)=\frac{r_{I}(t)}{p_{I}(t)}+\frac{\dot{p}_{I}(t)}{p_{I}(t)}-\frac{\dot{p}_{C}(t)}{p_{C}(t)}
\]

In our setting, given our choice of numeraire, we have $\dot{p}_{C}(t) / p_{C}(t)=0$. Moreover, $\dot{p}_{I}(t) / p_{I}(t)$ is given by (11.30). Finally,

\[
\frac{r_{I}(t)}{p_{I}(t)}=A-\delta
\]

given the linear technology in (11.28). Therefore, we have

\[
r_{C}(t)=A-\delta+\frac{\dot{p}_{I}(t)}{p_{I}(t)}
\]

and in steady state, from (11.30), the steady-state consumption-denominated rate of return is:

\[
r_{C}=A-\delta-(1-\alpha) g_{K}
\]

From (11.4), this implies a consumption growth rate of

\[
g_{C} \equiv \frac{\dot{C}(t)}{C(t)}=\frac{1}{\theta}\left(A-\delta-(1-\alpha) g_{K}-\rho\right)
\]

Finally, differentiate (11.27) and use the fact that labor is always constant to obtain

\[
\frac{\dot{C}(t)}{C(t)}=\alpha \frac{\dot{K}_{C}(t)}{K_{C}(t)}
\]

which, from the constancy of $\kappa(t)$ in steady state, implies the following steady-state relationship:

\[
g_{C}=\alpha g_{K}
\]

Substituting this into (11.32), we have

\[
g_{K}^{*}=\frac{A-\delta-\rho}{1-\alpha(1-\theta)}
\]

and

\[
g_{C}^{*}=\alpha \frac{A-\delta-\rho}{1-\alpha(1-\theta)}
\]

What about wages? Because labor is being used in the consumption good sector, there will be positive wages. Since labor markets are competitive, the wage rate at time $t$ is given by

\[
\wage(t)=(1-\alpha) p_{C}(t) B\left(\frac{(1-\kappa(t)) K(t)}{L}\right)^{\alpha}
\]

Therefore, in the balanced growth path,$ we obtain $\[
\begin{aligned}
\frac{\dot{w}(t)}{\wage(t)} & =\frac{\dot{p}_{C}(t)}{p_{C}(t)}+\alpha \frac{\dot{K}(t)}{K(t)} \\
& =\alpha g_{K}^{*}
\end{aligned}
\]

which implies that wages also grow at the same rate as consumption.\\
Moreover, with exactly the same arguments as in the previous section, it can be established that there are no transitional dynamics in this economy. This establishes the following result:
\begin{proposition}
In the above-described two-sector neoclassical economy, starting from any $K(0)>0$, consumption and labor income grow at the constant rate given by (11.34), while the capital stock grows at the constant rate (11.33).
\end{proposition}

It is straightforward to conduct policy analysis in this model, and as in the basic $AK$ model, taxes on investment income will depress growth. Similarly, a lower discount rate will increase the equilibrium growth rate of the economy

One important implication of this model, different from the neoclassical growth model, is that there is continuous capital deepening. Capital grows at a faster rate than consumption and output. Whether this is a realistic feature is debatable. The Kaldor facts, discussed above, include constant capital-output ratio as one of the requirements of balanced growth. Here we have steady state and "balanced growth" without this feature. For much of the 20th century, capital-output ratio has been constant, but it has been increasing steadily over the past 30 years. Part of the reason why it has been increasing recently but not before is because of relative price adjustments. New capital goods are of higher quality, and this needs to be incorporated in calculating the capital-output ratio. These calculations have only been performed in the recent past, which may explain why capital-output ratio has been constant in the earlier part of the century, but not recently.

\subsection{Growth with Externalities}
The model that started much of endogenous growth theory and revived economists' interest in economic growth was Paul Romer's (1986) paper. Romer's objective was to model the process of "knowledge accumulation". He realized that this would be difficult in the context of a competitive economy. His initial solution (later updated and improved in his and others' work during the 1990s) was to consider knowledge accumulation to be a byproduct of capital accumulation. In other words, Romer introduced technological spillovers, similar to those discussed in the context of human capital in Chapter 10. While arguably crude, this captures an important dimension of knowledge, that knowledge is a largely non-rival good-once a particular technology has been discovered, many firms can make use of this technology without preventing others using the same knowledge. Non-rivalry does not imply knowledge is also non-excludable (which would have made it a pure public good). A firm that discovers a new technology may use patents or trade secrecy to prevent others from using it, for example, in order to gain a competitive advantage. These issues will be discussed in the next part of the book. For now, it suffices to note that some of the important characteristics of "knowledge" and its role in the production process can be captured in a reduced-form way by introducing technological spillovers. We next discuss a version of the model in Romer's (1986) paper, which introduces such technological spillovers as an engine of economic growth. While the type of technological spillovers used in this model are unlikely to be important in practice, this model is a good starting point for our analysis of endogenous technological progress, since its similarity to the baseline $AK$ economy makes it a very tractable model of knowledge accumulation.\\
\subsubsection{Preferences and Technology}
Consider an economy without any population growth (we will see why this is important) and a production function with labor-augmenting knowledge (technology) that satisfies the standard assumptions, Assumptions 1 and 2. For reasons that will become clear, instead of working with the aggregate production function, let us assume that the production side of the economy consists of a set $[0,1]$ of firms. The production function facing each firm $i \in[0,1]$ is
\[
Y_{i}(t)=F\left(K_{i}(t), \Atech(t) L_{i}(t)\right)
\] where $K_i(t)$ and $L_i(t)$ are capital and labor rented by a firm $i$. Notice that $\Atech(t)$ is not indexed by $i$, since it is technology common to all firms. Let us normalize the measure of final good producers to 1 , so that we have the following market clearing conditions:
\[
\int_{0}^{1} K_{i}(t) d i=K(t)
\]

and

\[
\int_{0}^{1} L_{i}(t) d i=L
\]
where $L$ is the constant level of labor (supplied inelastically) in this economy. Firms are competitive in all markets, which implies that they will all hire the same capital to effective labor ratio, and moreover, factor prices will be given by their marginal products, thus
\begin{equation}
\begin{aligned}
\wage(t) & =\frac{\partial F(K(t), \Atech(t) L)}{\partial L} \\
R(t) & =\frac{\partial F(K(t), \Atech(t) L)}{\partial K(t)}
\end{aligned}
\end{equation}

The key assumption of Romer (1986) is that although firms take $\Atech(t)$ as given, this stock of technology (knowledge) advances endogenously for the economy as a whole. In particular, Romer assumes that this takes place because of spillovers across firms, and attributes spillovers to physical capital. Lucas (1988) develops a similar model in which the structure is identical, but spillovers work through human capital (i.e., while Romer has physical capital externalities, Lucas has human capital externalities).

The idea of externalities is not uncommon to economists, but both Romer and Lucas make an extreme assumption of sufficiently strong externalities such that $\Atech(t)$ can grow continuously at the economy level. In particular, Romer assumes

\[
\Atech(t)=B K(t)
\]

i.e., the knowledge stock of the economy is proportional to the capital stock of the economy. This can be motivated by "learning-by-doing" whereby, greater investments in certain sectors increases the experience (of firms, workers, managers) in the production process, making the production process itself more productive. Alternatively, the knowledge stock of the economy could be a function of the cumulative output that the economy has produced up to now, thus giving it more of a flavor of "learning-by-doing".

In any case, substituting for (11.36) into (11.35) and using the fact that all firms are functioning at the same capital-effective labor ratio,$ we obtain $the production function of the representative firm as

\[
Y(t)=F(K(t), B K(t) L) .
\]

Using the fact that $F(\cdot, \cdot)$ is homogeneous of degree 1 , we have

\[
\begin{aligned}
\frac{Y(t)}{K(t)} & =F(1, B L) \\
& =\tilde{f}(L)
\end{aligned}
\]

Output per capita can$ therefore $be written as:

\[
\begin{aligned}
y(t) & \equiv \frac{Y(t)}{L} \\
& =\frac{Y(t)}{K(t)} \frac{K(t)}{L} \\
& =k(t) \tilde{f}(L)
\end{aligned}
\]
 where again $k(t) \equiv K(t) / L$ is the capital-labor ratio in the economy.\\
As in the standard growth model, marginal products and factor prices can be expressed in terms of the normalized production function, now $\tilde{f}(L)$. In particular, we have

\[
\wage(t)=K(t) \tilde{f}^{\prime}(L)
\]

and

\[
R(t)=R=\tilde{f}(L)-L \tilde{f}^{\prime}(L)
\]

which is constant.\\
\subsubsection{Equilibrium}
An equilibrium is defined similarly to the neoclassical growth model, as a path of consumption and capital stock for the economy, $[C(t), K(t)]_{t=0}^{\infty}$ that maximize the utility of the representative household and wage and rental rates $[\wage(t), R(t)]_{t=0}^{\infty}$ that clear markets. The important feature is that because the knowledge spillovers, as specified in (11.36), are external to the firm, factor prices are given by (11.37) and (11.38) - that is, they do not price the role of the capital stock in increasing future productivity.

Since the market rate of return is $\rnet(t)=R(t)-\delta$, it is also constant. The usual consumer Euler equation (e.g., (11.4) above) then implies that consumption must grow at the constant rate,

\[
g_{C}^{*}=\frac{1}{\theta}\left(\tilde{f}(L)-L \tilde{f}^{\prime}(L)-\delta-\rho\right)
\]

It is also clear that capital grows exactly at the same rate as consumption, so the rate of capital, output and consumption growth are all given by $g_{C}^{*}$ as given by (11.39)-see Exercise 11.15.

Let us assume that

\[
\tilde{f}(L)-L \tilde{f}^{\prime}(L)-\delta-\rhoRate > 0
\]

so that there is positive growth, but also that growth is not fast enough to violate the transversality condition, in particular,

\[
(1-\theta)\left(\tilde{f}(L)-L \tilde{f}^{\prime}(L)-\delta\right)<\rho
\]

\begin{proposition}
Consider the above-described Romer model with physical capital externalities. Suppose that conditions (11.40) and (11.41) are satisfied. Then, there exists
a unique equilibrium path where starting with any level of capital stock $K(0)>0$, capital, output and consumption grow at the constant rate (11.39).
\end{proposition}
\begin{proof}
Much of this proposition is proved in the preceding discussion. You are asked to verify the transversality conditions and show that there are no transitional dynamics in Exercise 11.16.
\end{proof}

Population must be constant in this model because of the scale effect. Since $\tilde{f}(L)-L \tilde{f}^{\prime}(L)$ is always increasing in $L$ (by Assumption 1), a higher population (labor force) $L$ leads to a higher growth rate. The scale effect refers to this relationship between population and the equilibrium rate of economic growth. Now if population is growing, the economy will not admit a steady state and the growth rate of the economy will increase over time (output reaching infinity in finite time and violating the transversality condition). The implications of positive population growth are discussed further in Exercise 11.17. Scale effects and how they can be removed will be discussed in detail in Chapter 13.\\
\subsubsection{Pareto Optimal Allocations}
Given the presence of externalities, it is not surprising that the decentralized equilibrium characterized in Proposition 11.5 is not Pareto optimal. To characterize the allocation that maximizes the utility of the representative household, let us again set up on the current-value Hamiltonian. The per capita accumulation equation for this economy can be written as

\[
\dot{k}(t)=\tilde{f}(L) k(t)-c(t)-\delta k(t)
\]

The current-value Hamiltonian is

\[
\hat{H}(k, c, \mu)=\frac{c(t)^{1-\theta}-1}{1-\theta}+\mu[\tilde{f}(L) k(t)-c(t)-\delta k(t)]
\]

and has the necessary conditions:

\[
\begin{aligned}
\hat{H}_{c}(k, c, \mu) & =c(t)^{-\theta}-\mu(t)=0 \\
\hat{H}_{k}(k, c, \mu) & =\mu(t)[\tilde{f}(L)-\dep]=-\dot{\mu}(t)+\rho \mu(t), \\
\lim _{t \rightarrow \infty}[\exp(-\rhoRate t) \mu(t) k(t)] & =0 .
\end{aligned}
\]

These equations imply that the social planner's allocation will also have a constant growth rate for consumption (and output) given by

\[
g_{C}^{S}=\frac{1}{\theta}(\tilde{f}(L)-\delta-\rhoRate)
\]

which is always greater than $g_{C}^{*}$ as given by (11.39)-since $\tilde{f}(L)>\tilde{f}(L)-L \tilde{f}^{\prime}(L)$. Essentially, the social planner takes into account that by accumulating more capital, she is improving productivity in the future. Since this effect is external to the firms, the decentralized economy fails to internalize this externality. Therefore we have:

\begin{proposition}
In the above-described Romer model with physical capital externalities, the decentralized equilibrium is Pareto suboptimal and grows at a slower rate than the allocation that would maximize the utility of the representative household.
\end{proposition}

\subsection{Taking Stock}
This chapter ends our investigation of neoclassical growth models. It also opens the way for the analysis of endogenous technological progress in the next part of the book. The models presented in this chapter are, in many ways, more tractable and easier than those we have seen in earlier chapters. This is a feature of the linearity of the models (most clearly visible in the $AK$ model). This type of linearity removes transitional dynamics and leads to a more tractable mathematical structure. Linearity, of course, is an essential feature of any model that will exhibit sustained economic growth. If strong concavity sets in (especially concavity consistent with the Inada conditions as in Assumption 2), sustained growth will not be possible. Therefore, (asymptotic) linearity is an essential ingredient of any model that will lead to sustain growth. The baseline $AK$ model and its cousins make this linear structure quite explicit. While this type of linearity will be not as apparent (and often will be derived rather than assumed), it will also be a feature of the endogenous technology models studied in the next part of the book. Consequently, many of these endogenous technology models will be relatively tractable as well. Nevertheless, we will see that the linearity will often result from much more interesting economic interactions than being imposed in the aggregate production function of the economy. There is another sense in which the material in this chapter does not do justice to issues of sustained growth. As the discussion in Chapter 3 showed, modern economic growth is largely the result of technological progress. Except for the Romer model of Section 11.4, in the models studied in this chapter do not feature technological progress. This does not imply that they are necessarily inconsistent with the data. As our discussion in Chapter 3 indicated there is a lively debate about whether the observed total factor productivity growth is partly a result of mismeasurement of inputs. If this is the case, it could be that much of what we measure as technological progress is in fact capital deepening, which is the bread-and-butter of economic growth in the $AK$ model and its variants. Consequently, the debate about the measurement of total factor productivity has important implications for what types of models we should use for thinking about world economic growth and cross-country income differences.

The discussion in this chapter has also revealed another important tension. Chapters 3 and 8 demonstrated that the neoclassical growth model (or the simpler Solow growth model) have difficulty in generating the very large income differences across countries that\\
we observe in the data. Even if we choose quite large differences in cross-country distortions (for example, eightfold differences in effective tax rates), the implied steady-state differences in income per capita are relatively modest. We have seen that this has generated a large literature that seeks reasonable extensions of the neoclassical growth model in order to derive more elastic responses to policy distortions or other differences across countries. The models presented in this chapter, like those that we will encounter in the next part of the book, suffer from the opposite problem. They imply that even small differences in policies, technological opportunities or other characteristics of societies will lead to permanent differences in long-run growth rates. Consequently, these models can explain very large differences in living standards from small policy, institutional or technological differences. But this is both a blessing and a curse. Though capable of explaining large cross-country differences, these models also predict an ever expanding world distribution, since countries with different characteristics should grow at permanently different rates. The relative stability of the world income distribution in the postwar era is then a challenge to the baseline endogenous growth models. However, as we have seen, the world income distribution is not exactly stationary. While economists more sympathetic to the exogenous growth version of the neoclassical model emphasize the relative stability of the world income distribution, others see stratification and increased inequality. This debate can, in principle, be resolved by carefully mapping various types of endogenous growth theories to postwar data.

Nevertheless, there is more to understanding the nature of the growth process and the role of technological progress than simply looking at the postwar data. First, as illustrated in Chapter 1, the era of divergence is not the past 60 years, but the 19th century. Therefore, it is equally important to confront these models with historical data. Second, a major assumption of most endogenous growth models is that each country can be treated in isolation. This "each country as an island" approach is unlikely to be a good approximation to reality in most circumstances, and much less so when we endogenize technology. Most economies do not generate their own technology by R\&D or other processes, but partly import or adopt these technologies from more advanced nations (or from the world technology frontier). Consequently, a successful mapping of the theories to data requires us to enrich these theories and abandon the "each country as an island" assumption. We will do this later in the book both in the context of technology flows across countries and of international trade linkages. But the next part will follow the established literature and develop the models of endogenous technological progress without paying much attention to cross-country knowledge flows.

\subsection{References and Literature}
The $AK$ model is a special case of Rebelo's (1991), which was discussed in greater detail in Section 11.3 of this chapter. Solow's (1965) book also discussed the $AK$ model (naturally\\
with exogenous savings), but dismissed it as uninteresting. A more complete treatment of sustained neoclassical economic growth is provided in Jones and Manuelli (1990), who show that even convex models (with production function is that satisfy Assumption 1, but naturally not Assumption 2) are consistent with sustained long-run growth. Exercise 11.4 is a version of the convex neoclassical endogenous growth model of Jones and Manuelli.

Barro and Sala-i-Martin (2004) discuss a variety of two-sector endogenous growth models with physical and human capital, similar to the model presented in Section 11.2, though the model presented here is much simpler than similar ones analyzed in the literature.

Romer (1986) is the seminal paper of the endogenous growth literature and the model presented in Section 11.4 is based on this paper. Frankel (1962) analyzed a similar growth economy, but with exogenous constant saving rate. The importance of Romer's paper stems not only from the model itself, but from two other features. The first is its emphasis on potential non-competitive elements in order to generate long-run economic growth (in this case knowledge spillovers). The second is its emphasis on the non-rival nature of knowledge and ideas. These issues will be discussed in greater detail in the next part of the book.

Another paper that has played a major role in the new growth literature is Lucas (1988), which constructs an endogenous growth model similar to that of Romer (1986), but with human capital accumulation and human capital externalities. Lucas' model is also similar to the earlier contribution by Uzawa (1964). Lucas's paper has played two major roles in the literature. First, it emphasized the empirical importance of sustained economic growth and thus was instrumental in generating interest in the newly emerging endogenous growth models. Second, it emphasized the importance of human capital and especially of human capital externalities. Since the role of human capital was discussed extensively in Chapter 10, which also showed that the evidence for human capital externalities is rather limited, we focused on the Romer model rather than the Lucas model. It turns out that Lucas model also generates transitional dynamics, which are slightly more difficult to characterize than the standard neoclassical transitional dynamics. A version of the Lucas model is discussed in Exercise 11.20.

\subsection{Exercises}

\[
U(0)=\int_{0}^{\infty} \exp(-\rhoRate t) \frac{(c(t))^{1-\theta}-1}{1-\theta}
\]

with aggregate production function

\[
Y(t)=A K(t)+B L(t)
\]
 where A, B>0$.\\
(1) Define a competitive equilibrium for this economy.\\
(2) Set up the current-value Hamiltonian for an individual and characterize the necessary conditions for consumer maximization. Combine these with equilibrium factor market prices and derive the equilibrium path. Show that the equilibrium path displays non-trivial transitional dynamics.\\
(3) Determine the evolution of the labor share of national income over time.\\
(4) Analyze the impact of an unanticipated increase in $B$ on the equilibrium path.\\
(5) Prove that the equilibrium is Pareto optimal.

\[
U(0)=\int_{0}^{\infty} \exp(-\rhoRate t) \frac{(c(t))^{1-\theta}-1}{1-\theta}
\]

with production function

\[
Y(t)=A\left[L(t)^{\frac{\sigma-1}{\sigma}}+K(t)^{\frac{\sigma-1}{\sigma}}\right]^{\frac{\sigma}{\sigma-1}}
\]

(1) Define a competitive equilibrium for this economy.\\
(2) Set up the current-value Hamiltonian for an individual and characterize the necessary conditions for consumer maximization. Combine these with equilibrium factor market prices and derive the equilibrium path.\\
(3) Prove that the equilibrium is Pareto optimal in this case.\\
(4) Show that if  \sigma \leq 1$, sustained growth is not possible.\\
(5) Show that if $A$ and $\sigma$ are sufficiently high, this model generates asymptotically sustained growth due to capital accumulation. Interpret this result.\\
(6) Characterize the transitional dynamics of the equilibrium path.\\
(7) What is happening to the share of capital in national income? Is this plausible? How would you modify the model to make sure that the share of capital in national income remains constant?\\
(8) Now assume that returns from capital are taxed at the rate $\tau$. Determine the asymptotic growth rate of consumption and output.

Suppose that the first country has a capital income tax rate of $\tau=0.2$, while the second country has a tax rate of $\tau^{\prime}=0.4$. Suppose that the two countries start with the same level of income in 1900 and experience no change in technology or policies for the next 100 years. What will be the relative income gap between the two countries in the year 2000? Discuss this result and explain why you do (or do not) find the implications plausible.\\

\[
I(t)=A\left(K_{I}(t)\right)^{\beta}\left(L_{I}(t)\right)^{1-\beta}
\]
 where $\beta \in(\alpha, 1)$. The labor market clearing condition requires $L_{C}(t)+L_{I}(t) \leq L(t)$. The rest of the environment is unchanged.\\
(1) Define a competitive equilibrium.\\
(2) Characterize the steady-state equilibrium and show that it does not involve sustained growth.\\
(3) Explain why the long-run growth implications of this model differ from those of Section 11.3.\\
(4) Analyze the steady-state income differences between two economies taxing capital at the rates $\tau$ and $\tau'$. What are the roles of the parameters $\alpha$ and $\beta$ in determining these relative differences? Why do the implied magnitudes differ from those in the one-sector neoclassical growth model?

\[
U(0)=\sum_{t=0}^{\infty} \beta^{t}\left[\frac{C(t)^{1-\theta}-1}{1-\theta}\right]
\]
 where $C(t)$ is consumption, and $\beta \in(0,1)$. Total population is equal to $L$ and there is no population growth and labor is supplied inelastically. The production side of the economy consists of a continuum 1 of firms, each with production function

\[
Y_{i}(t)=F\left(K_{i}(t), \Atech(t) L_{i}(t)\right)
\]
 where $L_{i}(t)$ is employment of firm $i$ at time $t, K_{i}(t)$ is capital used by firm $i$ at time $t$, and $\Atech(t)$ is a common technology term. Market clearing implies that $\int_{0}^{1} K_{i}(t) d i=K(t) , where $K(t)$ is the total capital stock at time $t$, and $\int_{0}^{1} L_{i}(t) d i=L(t)$. Assume that capital fully depreciates, so that the resource constraint of the economy is

\[
K(t+1)=\int_{0}^{1} Y_{i}(t) d i-C(t)
\]

Assume also that labor-augmenting productivity at time $t, \Atech(t)$, is given by

\[
\Atech(t)=K(t)
\]

(1) Explain (11.42) and why it implies a (non-pecuniary) externality.\\
(2) Define a competitive equilibrium (where all agents are price takers-but naturally not all markets are complete).\\
(3) Show that there exists a unique balanced growth path competitive equilibrium, where the economy grows (or shrinks) at a constant rate every period. Provide a\\
condition on $F, \beta$ and $\theta$$ such that $this growth rate is positive, but the transversality condition is still satisfied.\\
(4) Argue (without providing the math) why any equilibrium must be along the balanced growth path characterized in part 3 at all points.\\
(5) Is this a good model of endogenous growth? If yes, explain why. If not, contrast it with what you consider to be better models.

\[
\int_{0}^{\infty} \exp(-\rhoRate t) \frac{C(t)^{1-\theta}-1}{1-\theta} d t
\]
 where C(t)$ is consumption of the final good, which is produced as

\[
Y(t)=A K(t)^{\alpha} H_{P}^{1-\alpha}(t)
\]

where $K(t)$ is capital and $H(t)$ is human capital, and $H_{P}(t)$ denotes human capital used in production. The accumulation equations are as follows:

\[
\dot{K}(t)=I(t)-\dep K(t)
\]

for capital and

\[
\dot{H}(t)=B H_{E}(t)-\delta H(t)
\]
 where $H_{E}(t)$ is human capital devoted to education (further human capital accumulation), and the depreciation of human capital is assumed to be at the same rate as physical capital for simplicity $(\delta)$. The resource constraints of the economy are

\[
I(t)+C(t) \leq Y(t)
\]

and

\[
H_{E}(t)+H_{P}(t) \leq H(t)
\]

(1) Interpret the second resource constraint.\\
(2) Denote the fraction of human capital allocated to production by $\phi(t)$, and calculate the growth rate of final output as a function of $\phi(t)$ and the growth rates of accumulable factors.\\
(3) Assume that $\phi(t)$ is constant, and characterize the balanced growth path of the economy (with constant interest rate and constant rate of growth for capital and output). Show that in this balanced growth path, we have $r^{*} \equiv B-\delta$ and the growth rate of consumption, capital, human capital and output are given by $g^{*} \equiv (B-\delta-\rhoRate) / \theta$. Show also that there exists a unique value of $k^{*} \equiv K / H$ consistent with balanced growth path.\\
(4) Determine the parameter restrictions to make sure that the transversality condition is satisfied.\\
(5) Now analyze the transitional dynamics of the economy starting with $K / H$ different from $k^*$ [Hint: look at dynamics in three variables, $k \equiv K / H, \chi \equiv C / K$ and $\phi$, and consider the cases $\alpha<\theta$ and $\alpha \geq \theta$ separately].
\end{document}
